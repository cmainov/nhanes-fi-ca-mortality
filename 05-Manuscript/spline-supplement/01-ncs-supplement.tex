\documentclass{article}
\usepackage[svgnames]{xcolor}
\usepackage{listings}


\lstset{language=R,
    basicstyle=\small\ttfamily,
    stringstyle=\color{DarkGreen},
    otherkeywords={0,1,2,3,4,5,6,7,8,9},
    morekeywords={TRUE,FALSE},
    deletekeywords={data,frame,length,as,character},
    keywordstyle=\color{blue},
    commentstyle=\color{DarkGreen},
}


%% Citations/bibtex
\usepackage[style=nature,%
            autocite=footnote,%
             backend=bibtex,%
             doi=true,% include doi in citation
             firstinits=true,%. %% abbreviates first name to just letters and not full first name
             ]{biblatex}
\addbibresource{01-ncs-supplement.bib}

% customize the bibliography citations by removing some fields
\AtEveryBibitem{%
  \clearfield{issn}%
  \clearfield{url}%
  \clearfield{urlyear}% remove url access year from citation
  \clearfield{pagetotal}%
}
%% math sybmols
\usepackage{amsmath}


\title{Specifying a Natural Cubic Spline}
\author{Christian A. Maino Vieytes}
\date{May 9, 2023}

\begin{document}
\maketitle
\pagenumbering{gobble}
\pagenumbering{arabic}
Using splines in regression contexts has become a popular method for flexibly modeling exposure-outcome relationships in epidemiological studies \supercite{greenland_dose-response_1995, witte_nested_1997}



\newpage
%% R code chunk %%
\begin{lstlisting}
#Calculate CE for each counterparty
Value.A <- data.frame() #MTM value of each contract within cp A
# ...
for(i in 1:length(foo)){
  if(isTrue(as.character(portfolio_data[i,1])=="A")==TRUE){
    # ...
  }
}
\end{lstlisting}
%% end R code chunk %%


% print bibliography
\printbibliography[type=article,title={References}]
\end{document}